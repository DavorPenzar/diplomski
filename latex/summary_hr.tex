\begin{sazetak}
    Proučavajući teorijski svojstvene vrijednosti i svojstvene funkcije Laplaceovog operatora uočena su obilježja kojima se njihovo računanje može pojednostaviti. Prvenstveno, one ne ovise o afinim izometričnim transformacijama skupova (osim što se domena svojstvenih funkcija, očito, izometrično transformira). Također, skaliranjem skupa svojstvene se vrijednosti skaliraju multiplikativnim inverzom kvadrata skalara, dok se svojstvene funkcije komponiraju zdesna sa skalarom. Nadalje, one neprekidno ovise o domeni i svojstvene su vrijednosti padajuće s obzirom na relaciju inkluzije. Iz svega se navedenog zaključeno je da se određene domene u svrhu numeričkog računanja svojstvenih vrijednosti i svojstvenih funkcija Laplaceovog operatora (to jest, najmanje svojstvene vrijednosti u slučaju ovog rada) mogu aproksimirati \emph{jednostavnijim}, \emph{dovoljno sličnim} i normaliziranim domenama.

    \par

    Kao jedan tip jednostavnijih domena koje mogu poprimiti \emph{razne} oblike odabrani su poligoni. Zbog opaženih obilježja svojstvenih vrijednosti i svojstvenih funkcija Laplaceovog operatora, u svrhu njihova računanja modelom strojnog učenja bilo je poželjno poligone opisati karakterizacijama koje su dovoljno diskriminirajuće, ali i robusne na određene transformacije. U radu su ponuđene nude tri različite karakterizacije poligona, a u praktičnom nastavku rada proučavani su samo trokuti kao prototipi poligona.

    \par

    Za svaku od karakterizacija, nakon proučavanja literature i eksploratorne analize izrađenog skupa podataka, odabrana je jedna pogodna metoda strojnog učenja za početni problem. Tako su konstruirani dva modela linearne regresije, neuronska mreža i konvolucijska neuronska mreža. Nakon treniranja modela njihove su uspješnosti ispitane na testnim skupovima podataka i modeli su uspoređeni s obzirom na uspješnosti i brzine. Rezultati su u većini slučajeva bili obećavajući, ali generalizabilnost modela na složenije poligone nije potvrđena jer model koji se najjednostavnije može generalizirati---konvolucijska neuronska mreža---luči najlošije rezultate.

    \par
\end{sazetak}
